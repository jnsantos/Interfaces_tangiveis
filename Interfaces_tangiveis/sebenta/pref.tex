\chapter{Prefácio}

Este documento, tentativamente chamado "Sebenta de Interfaces Tangíveis" procura dotar o seu leitor ou aquele que rapidamente o ler na diagonal, de elementos necessários para compreender e ser capaz de acompanhar as aulas.

Obviamente que se deverá referir que este documento não invalida a presença em sala de aula. Apesar de ter tentado verter aqui o máximo possível de informação que julgo ser pertinente, haverá sempre algo \textit{extra} em sala de aula, fruto da partilha, da experimentação, do acaso

Resta-me dizer que nesta cadeira, e apesar da variedade de materiais que irá utilizar, o conceito que nos deverá guiar será o de que deveremos tentar sempre divertirmo-nos e experimentar. E que, quando falharmos, deveremos abraçar e aprender com o erro e não pensar nele como algo negativo, mas sim como uma lição. Ou, como Thomas Edison disse: 
\begin{quotation}
	
	
	Eu não falhei. Eu descobri 10.000 maneiras que não funcionam.
	
\end{quotation}